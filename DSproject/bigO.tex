\documentclass[]{article}
\usepackage{lmodern}
\usepackage{amssymb,amsmath}
\usepackage{ifxetex,ifluatex}
\usepackage{fixltx2e} % provides \textsubscript
\ifnum 0\ifxetex 1\fi\ifluatex 1\fi=0 % if pdftex
  \usepackage[T1]{fontenc}
  \usepackage[utf8]{inputenc}
\else % if luatex or xelatex
  \ifxetex
    \usepackage{mathspec}
  \else
    \usepackage{fontspec}
  \fi
  \defaultfontfeatures{Ligatures=TeX,Scale=MatchLowercase}
\fi
% use upquote if available, for straight quotes in verbatim environments
\IfFileExists{upquote.sty}{\usepackage{upquote}}{}
% use microtype if available
\IfFileExists{microtype.sty}{%
\usepackage{microtype}
\UseMicrotypeSet[protrusion]{basicmath} % disable protrusion for tt fonts
}{}
\usepackage[margin=1in]{geometry}
\usepackage{hyperref}
\hypersetup{unicode=true,
            pdfborder={0 0 0},
            breaklinks=true}
\urlstyle{same}  % don't use monospace font for urls
\usepackage{color}
\usepackage{fancyvrb}
\newcommand{\VerbBar}{|}
\newcommand{\VERB}{\Verb[commandchars=\\\{\}]}
\DefineVerbatimEnvironment{Highlighting}{Verbatim}{commandchars=\\\{\}}
% Add ',fontsize=\small' for more characters per line
\usepackage{framed}
\definecolor{shadecolor}{RGB}{248,248,248}
\newenvironment{Shaded}{\begin{snugshade}}{\end{snugshade}}
\newcommand{\KeywordTok}[1]{\textcolor[rgb]{0.13,0.29,0.53}{\textbf{#1}}}
\newcommand{\DataTypeTok}[1]{\textcolor[rgb]{0.13,0.29,0.53}{#1}}
\newcommand{\DecValTok}[1]{\textcolor[rgb]{0.00,0.00,0.81}{#1}}
\newcommand{\BaseNTok}[1]{\textcolor[rgb]{0.00,0.00,0.81}{#1}}
\newcommand{\FloatTok}[1]{\textcolor[rgb]{0.00,0.00,0.81}{#1}}
\newcommand{\ConstantTok}[1]{\textcolor[rgb]{0.00,0.00,0.00}{#1}}
\newcommand{\CharTok}[1]{\textcolor[rgb]{0.31,0.60,0.02}{#1}}
\newcommand{\SpecialCharTok}[1]{\textcolor[rgb]{0.00,0.00,0.00}{#1}}
\newcommand{\StringTok}[1]{\textcolor[rgb]{0.31,0.60,0.02}{#1}}
\newcommand{\VerbatimStringTok}[1]{\textcolor[rgb]{0.31,0.60,0.02}{#1}}
\newcommand{\SpecialStringTok}[1]{\textcolor[rgb]{0.31,0.60,0.02}{#1}}
\newcommand{\ImportTok}[1]{#1}
\newcommand{\CommentTok}[1]{\textcolor[rgb]{0.56,0.35,0.01}{\textit{#1}}}
\newcommand{\DocumentationTok}[1]{\textcolor[rgb]{0.56,0.35,0.01}{\textbf{\textit{#1}}}}
\newcommand{\AnnotationTok}[1]{\textcolor[rgb]{0.56,0.35,0.01}{\textbf{\textit{#1}}}}
\newcommand{\CommentVarTok}[1]{\textcolor[rgb]{0.56,0.35,0.01}{\textbf{\textit{#1}}}}
\newcommand{\OtherTok}[1]{\textcolor[rgb]{0.56,0.35,0.01}{#1}}
\newcommand{\FunctionTok}[1]{\textcolor[rgb]{0.00,0.00,0.00}{#1}}
\newcommand{\VariableTok}[1]{\textcolor[rgb]{0.00,0.00,0.00}{#1}}
\newcommand{\ControlFlowTok}[1]{\textcolor[rgb]{0.13,0.29,0.53}{\textbf{#1}}}
\newcommand{\OperatorTok}[1]{\textcolor[rgb]{0.81,0.36,0.00}{\textbf{#1}}}
\newcommand{\BuiltInTok}[1]{#1}
\newcommand{\ExtensionTok}[1]{#1}
\newcommand{\PreprocessorTok}[1]{\textcolor[rgb]{0.56,0.35,0.01}{\textit{#1}}}
\newcommand{\AttributeTok}[1]{\textcolor[rgb]{0.77,0.63,0.00}{#1}}
\newcommand{\RegionMarkerTok}[1]{#1}
\newcommand{\InformationTok}[1]{\textcolor[rgb]{0.56,0.35,0.01}{\textbf{\textit{#1}}}}
\newcommand{\WarningTok}[1]{\textcolor[rgb]{0.56,0.35,0.01}{\textbf{\textit{#1}}}}
\newcommand{\AlertTok}[1]{\textcolor[rgb]{0.94,0.16,0.16}{#1}}
\newcommand{\ErrorTok}[1]{\textcolor[rgb]{0.64,0.00,0.00}{\textbf{#1}}}
\newcommand{\NormalTok}[1]{#1}
\usepackage{graphicx,grffile}
\makeatletter
\def\maxwidth{\ifdim\Gin@nat@width>\linewidth\linewidth\else\Gin@nat@width\fi}
\def\maxheight{\ifdim\Gin@nat@height>\textheight\textheight\else\Gin@nat@height\fi}
\makeatother
% Scale images if necessary, so that they will not overflow the page
% margins by default, and it is still possible to overwrite the defaults
% using explicit options in \includegraphics[width, height, ...]{}
\setkeys{Gin}{width=\maxwidth,height=\maxheight,keepaspectratio}
\IfFileExists{parskip.sty}{%
\usepackage{parskip}
}{% else
\setlength{\parindent}{0pt}
\setlength{\parskip}{6pt plus 2pt minus 1pt}
}
\setlength{\emergencystretch}{3em}  % prevent overfull lines
\providecommand{\tightlist}{%
  \setlength{\itemsep}{0pt}\setlength{\parskip}{0pt}}
\setcounter{secnumdepth}{0}
% Redefines (sub)paragraphs to behave more like sections
\ifx\paragraph\undefined\else
\let\oldparagraph\paragraph
\renewcommand{\paragraph}[1]{\oldparagraph{#1}\mbox{}}
\fi
\ifx\subparagraph\undefined\else
\let\oldsubparagraph\subparagraph
\renewcommand{\subparagraph}[1]{\oldsubparagraph{#1}\mbox{}}
\fi

%%% Use protect on footnotes to avoid problems with footnotes in titles
\let\rmarkdownfootnote\footnote%
\def\footnote{\protect\rmarkdownfootnote}

%%% Change title format to be more compact
\usepackage{titling}

% Create subtitle command for use in maketitle
\providecommand{\subtitle}[1]{
  \posttitle{
    \begin{center}\large#1\end{center}
    }
}

\setlength{\droptitle}{-2em}

  \title{}
    \pretitle{\vspace{\droptitle}}
  \posttitle{}
    \author{}
    \preauthor{}\postauthor{}
    \date{}
    \predate{}\postdate{}
  

\begin{document}

\section{Time Complexity of
Algorithms}\label{time-complexity-of-algorithms}

let's assume that we all know what is the subject of this presentation.
In the following we are going to watch how deciding on type of our
algorithm can make difference in run time of our algorithm. I suggest if
you don't know the main idea just google the main title of this page.

\paragraph{first let's see a complexity
chart}\label{first-lets-see-a-complexity-chart}

I wanted to hide the code for creating this plot but sometimes I don't
feel like being humble. so :

\begin{Shaded}
\begin{Highlighting}[]
\KeywordTok{library}\NormalTok{(ggplot2) ; }\KeywordTok{library}\NormalTok{(RColorBrewer)}
\NormalTok{xd <-}\StringTok{ }\KeywordTok{data.frame}\NormalTok{(}\DataTypeTok{x =} \KeywordTok{c}\NormalTok{(}\FloatTok{0.01}\NormalTok{, }\DecValTok{15}\NormalTok{))}
\NormalTok{cl <-}\StringTok{ }\KeywordTok{brewer.pal}\NormalTok{(}\DataTypeTok{n =} \DecValTok{6}\NormalTok{,}\DataTypeTok{name =} \StringTok{'Set2'}\NormalTok{)}

\NormalTok{g <-}\StringTok{ }\KeywordTok{ggplot}\NormalTok{(}\DataTypeTok{data =}\NormalTok{ xd, }\KeywordTok{aes}\NormalTok{(}\DataTypeTok{x =}\NormalTok{ x))}
\NormalTok{g <-}\StringTok{ }\NormalTok{g }\OperatorTok{+}\StringTok{ }\KeywordTok{stat_function}\NormalTok{(}\DataTypeTok{fun =}\NormalTok{ n , }\DataTypeTok{size =} \FloatTok{1.5}\NormalTok{, }\KeywordTok{aes}\NormalTok{(}\DataTypeTok{color =}\NormalTok{ cl[}\DecValTok{1}\NormalTok{])) }\OperatorTok{+}
\StringTok{      }\KeywordTok{stat_function}\NormalTok{(}\DataTypeTok{fun =}\NormalTok{ logn , }\DataTypeTok{size =} \FloatTok{1.5}\NormalTok{, }\KeywordTok{aes}\NormalTok{(}\DataTypeTok{color =}\NormalTok{ cl[}\DecValTok{2}\NormalTok{])) }\OperatorTok{+}
\StringTok{      }\KeywordTok{stat_function}\NormalTok{(}\DataTypeTok{fun =}\NormalTok{ nlogn , }\DataTypeTok{size =} \FloatTok{1.5}\NormalTok{, }\KeywordTok{aes}\NormalTok{(}\DataTypeTok{color =}\NormalTok{ cl[}\DecValTok{3}\NormalTok{])) }\OperatorTok{+}
\StringTok{      }\KeywordTok{stat_function}\NormalTok{(}\DataTypeTok{fun =}\NormalTok{ n2, }\DataTypeTok{size =} \FloatTok{1.5}\NormalTok{, }\KeywordTok{aes}\NormalTok{(}\DataTypeTok{color =}\NormalTok{ cl[}\DecValTok{4}\NormalTok{])) }\OperatorTok{+}
\StringTok{      }\KeywordTok{stat_function}\NormalTok{(}\DataTypeTok{fun =}\NormalTok{ pow2n , }\DataTypeTok{size =} \FloatTok{1.5}\NormalTok{, }\KeywordTok{aes}\NormalTok{(}\DataTypeTok{color =}\NormalTok{ cl[}\DecValTok{5}\NormalTok{])) }\OperatorTok{+}
\StringTok{      }\KeywordTok{stat_function}\NormalTok{(}\DataTypeTok{fun =}\NormalTok{ factn , }\DataTypeTok{size =} \FloatTok{1.5}\NormalTok{, }\KeywordTok{aes}\NormalTok{(}\DataTypeTok{color =}\NormalTok{ cl[}\DecValTok{6}\NormalTok{]))}
\NormalTok{g <-}\StringTok{ }\NormalTok{g }\OperatorTok{+}\StringTok{ }\KeywordTok{scale_color_identity}\NormalTok{(}
      \DataTypeTok{name =} \StringTok{'complexity Chart'}\NormalTok{,}
      \DataTypeTok{breaks =}\NormalTok{ cl,}
      \DataTypeTok{labels =} \KeywordTok{c}\NormalTok{(}\StringTok{'n'}\NormalTok{, }\StringTok{'log(n)'}\NormalTok{, }\StringTok{'n*log(n)'}\NormalTok{, }\StringTok{'n^2'}\NormalTok{, }\StringTok{'2^n'}\NormalTok{, }\StringTok{'n!'}\NormalTok{),}
      \DataTypeTok{guide =} \StringTok{'legend'}
\NormalTok{)}
\NormalTok{g <-}\StringTok{ }\NormalTok{g }\OperatorTok{+}\StringTok{ }\KeywordTok{ylim}\NormalTok{(}\KeywordTok{c}\NormalTok{(}\OperatorTok{-}\DecValTok{10}\NormalTok{, }\DecValTok{200}\NormalTok{))}
\NormalTok{g <-}\StringTok{ }\NormalTok{g }\OperatorTok{+}\StringTok{ }\KeywordTok{labs}\NormalTok{(}\DataTypeTok{x =} \StringTok{'Elements'}\NormalTok{,}\DataTypeTok{y =}\StringTok{'Operations'}\NormalTok{)}
\NormalTok{g}
\end{Highlighting}
\end{Shaded}

\section[]{\texorpdfstring{\protect\includegraphics{bigO_files/figure-latex/unnamed-chunk-2-1.pdf}}{}}\label{section}

\section{Log(n) ; Binary Search}\label{logn-binary-search}

in this section an algorithm with Big-O of level log(n) will be
explained and simulated.

binary search is one of the fastest algorithms which has some
assumptions. let's take a look at this GIF :
\includegraphics{./bins.gif}

I believe this image prepared us for jumping into codeing section but
before that let's declare or assumptions:

\begin{itemize}
\item
  the array we pass to our function \emph{must} be \textbf{sorted} in an
  ascending way. why? think about we we're going to measure
\item
  another assumption is our target element is inside the array
\end{itemize}

\begin{Shaded}
\begin{Highlighting}[]
\NormalTok{bin_search <-}\StringTok{ }\ControlFlowTok{function}\NormalTok{ (v, t) \{}
\NormalTok{      lo <-}\StringTok{ }\DecValTok{1}
\NormalTok{      hi <-}\StringTok{ }\KeywordTok{length}\NormalTok{(v)}
      \ControlFlowTok{while}\NormalTok{ (lo }\OperatorTok{<=}\StringTok{ }\NormalTok{hi) \{}
\NormalTok{            mid =}\StringTok{ }\KeywordTok{round}\NormalTok{((lo }\OperatorTok{+}\StringTok{ }\NormalTok{hi) }\OperatorTok{/}\StringTok{ }\DecValTok{2}\NormalTok{)}
            
            \ControlFlowTok{if}\NormalTok{ (}\KeywordTok{abs}\NormalTok{(v[mid] }\OperatorTok{-}\StringTok{ }\NormalTok{t) }\OperatorTok{<=}\StringTok{ }\DecValTok{0}\NormalTok{ ) \{}
                  \KeywordTok{return}\NormalTok{(mid)}
                  \ControlFlowTok{break}
\NormalTok{            \} }\ControlFlowTok{else} \ControlFlowTok{if}\NormalTok{ (v[mid] }\OperatorTok{<}\StringTok{ }\NormalTok{t) \{}
\NormalTok{                  lo =}\StringTok{ }\NormalTok{mid }\OperatorTok{+}\StringTok{ }\DecValTok{1}
\NormalTok{            \} }\ControlFlowTok{else}\NormalTok{ \{}
\NormalTok{                  hi =}\StringTok{ }\NormalTok{mid }\OperatorTok{-}\StringTok{ }\DecValTok{1}
\NormalTok{            \}}
\NormalTok{      \}}
\NormalTok{\}}
\end{Highlighting}
\end{Shaded}

let's test this function with an array of random generated numbers ,
huh? :

\begin{Shaded}
\begin{Highlighting}[]
\NormalTok{test_array <-}\StringTok{ }\KeywordTok{seq}\NormalTok{(}\DataTypeTok{from =} \DecValTok{7}\NormalTok{,}\DataTypeTok{to =} \DecValTok{2100}\NormalTok{,}\DataTypeTok{by =} \DecValTok{7}\NormalTok{)}
\NormalTok{test_array[}\DecValTok{276}\NormalTok{]}
\end{Highlighting}
\end{Shaded}

\begin{verbatim}
## [1] 1932
\end{verbatim}

so we want to find out whether the function outputs 276 when we pass
1932 to it or not. we'll see:

\begin{Shaded}
\begin{Highlighting}[]
\KeywordTok{bin_search}\NormalTok{(}\DataTypeTok{v =}\NormalTok{ test_array,}\DataTypeTok{t =} \DecValTok{1932}\NormalTok{)}
\end{Highlighting}
\end{Shaded}

\begin{verbatim}
## [1] 276
\end{verbatim}

looks like our function works properly. now it's time for

\section{Simulation}\label{simulation}

in order to report a rather precise number I couldn't use R language's
built in function which is named \texttt{\{r\}\ Sys.time()} so I went
through an advanced package named microbenchmark. I used \textbf{uniform
distribution} for creating random numbers and I ran my time measuring
for 200 times in order to have more coherent data.

\begin{Shaded}
\begin{Highlighting}[]
\KeywordTok{library}\NormalTok{(microbenchmark)}
\KeywordTok{set.seed}\NormalTok{(}\DecValTok{12345}\NormalTok{)}
\NormalTok{df <-}\StringTok{ }\KeywordTok{microbenchmark}\NormalTok{(}
\DataTypeTok{bin2   =}  \KeywordTok{bin_search}\NormalTok{(}\DataTypeTok{v =} \KeywordTok{seq}\NormalTok{(}\DecValTok{1}\NormalTok{,}\DecValTok{100}\NormalTok{),}\DataTypeTok{t =} \KeywordTok{round}\NormalTok{(}\KeywordTok{runif}\NormalTok{(}\DecValTok{1}\NormalTok{,}\DataTypeTok{min =} \DecValTok{1}\NormalTok{,}\DataTypeTok{max =} \DecValTok{100}\NormalTok{))),}
\DataTypeTok{bin3 =}  \KeywordTok{bin_search}\NormalTok{(}\DataTypeTok{v =} \KeywordTok{seq}\NormalTok{(}\DecValTok{1}\NormalTok{,}\DecValTok{1000}\NormalTok{),}\DataTypeTok{t =} \KeywordTok{round}\NormalTok{(}\KeywordTok{runif}\NormalTok{(}\DecValTok{1}\NormalTok{,}\DataTypeTok{min =} \DecValTok{1}\NormalTok{,}\DataTypeTok{max =} \DecValTok{500}\NormalTok{))),}
\DataTypeTok{bin4   =}  \KeywordTok{bin_search}\NormalTok{(}\DataTypeTok{v =} \KeywordTok{seq}\NormalTok{(}\DecValTok{1}\NormalTok{,}\DecValTok{10000}\NormalTok{),}\DataTypeTok{t =} \KeywordTok{round}\NormalTok{(}\KeywordTok{runif}\NormalTok{(}\DecValTok{1}\NormalTok{,}\DataTypeTok{min =} \DecValTok{1}\NormalTok{,}\DataTypeTok{max =} \DecValTok{1000}\NormalTok{))),}
\DataTypeTok{bin5 =}  \KeywordTok{bin_search}\NormalTok{(}\DataTypeTok{v =} \KeywordTok{seq}\NormalTok{(}\DecValTok{1}\NormalTok{,}\DecValTok{100000}\NormalTok{),}\DataTypeTok{t =} \KeywordTok{round}\NormalTok{(}\KeywordTok{runif}\NormalTok{(}\DecValTok{1}\NormalTok{,}\DataTypeTok{min =} \DecValTok{1}\NormalTok{,}\DataTypeTok{max =} \DecValTok{5000}\NormalTok{))),}
\DataTypeTok{bin6   =}  \KeywordTok{bin_search}\NormalTok{(}\DataTypeTok{v =} \KeywordTok{seq}\NormalTok{(}\DecValTok{1}\NormalTok{,}\DecValTok{1000000}\NormalTok{),}\DataTypeTok{t =} \KeywordTok{round}\NormalTok{(}\KeywordTok{runif}\NormalTok{(}\DecValTok{1}\NormalTok{,}\DataTypeTok{min =} \DecValTok{1}\NormalTok{,}\DataTypeTok{max =} \DecValTok{10000}\NormalTok{))),}
\DataTypeTok{bin7   =}  \KeywordTok{bin_search}\NormalTok{(}\DataTypeTok{v =} \KeywordTok{seq}\NormalTok{(}\DecValTok{1}\NormalTok{,}\DecValTok{10000000}\NormalTok{),}\DataTypeTok{t =} \KeywordTok{round}\NormalTok{(}\KeywordTok{runif}\NormalTok{(}\DecValTok{1}\NormalTok{,}\DataTypeTok{min =} \DecValTok{1}\NormalTok{,}\DataTypeTok{max =} \DecValTok{100000}\NormalTok{))),}
\DataTypeTok{times =} \DecValTok{200}\NormalTok{)}

\NormalTok{df}
\end{Highlighting}
\end{Shaded}

\begin{verbatim}
## Unit: microseconds
##  expr    min     lq     mean  median     uq      max neval
##  bin2 10.573 13.971 23.62385 15.8590 25.109  240.142   200
##  bin3  8.684 16.614 38.63841 18.5020 30.018 2256.799   200
##  bin4 15.859 19.257 29.12337 26.0540 33.794  201.251   200
##  bin5 17.747 21.901 42.59546 23.7890 39.269  864.660   200
##  bin6 17.369 24.921 37.84739 26.4310 40.968  368.519   200
##  bin7 20.768 27.564 42.04795 29.6405 45.499  308.106   200
\end{verbatim}

and finally plots of these numbers :

\begin{Shaded}
\begin{Highlighting}[]
\NormalTok{cl2 <-}\StringTok{ }\KeywordTok{brewer.pal}\NormalTok{(}\DataTypeTok{name =} \StringTok{'Accent'}\NormalTok{, }\DataTypeTok{n =}\DecValTok{6}\NormalTok{)}
\KeywordTok{boxplot}\NormalTok{(df,}\DataTypeTok{col =}\NormalTok{ cl2,}\DataTypeTok{log =}\NormalTok{ T,}\DataTypeTok{outline =}\NormalTok{ F,}\DataTypeTok{ylime =} \KeywordTok{c}\NormalTok{(}\DecValTok{10}\NormalTok{,}\DecValTok{55}\NormalTok{))}
\end{Highlighting}
\end{Shaded}

\includegraphics{bigO_files/figure-latex/unnamed-chunk-6-1.pdf}

\begin{Shaded}
\begin{Highlighting}[]
\KeywordTok{autoplot}\NormalTok{(df)}
\end{Highlighting}
\end{Shaded}

\begin{verbatim}
## Coordinate system already present. Adding new coordinate system, which will replace the existing one.
\end{verbatim}

\includegraphics{bigO_files/figure-latex/unnamed-chunk-7-1.pdf}


\end{document}
